\usepackage[utf8]{inputenc}

% use packages - sort of like include from C++ or import from python
% Some of them are automatic others just add new commands
\usepackage[T1]{fontenc}
\usepackage{enumitem}
\usepackage{listings}
\usepackage{geometry}
\usepackage{tabularx}
\usepackage{fancyhdr}
\usepackage{booktabs}
\usepackage{graphicx}
\usepackage{parskip}
\usepackage{longtable}
\usepackage{array}
\usepackage{lastpage}
\usepackage{csquotes}
\usepackage{hyperref}
\usepackage{listings}
\usepackage{color}
\usepackage[dvipsnames]{xcolor}
\usepackage{textcomp}
\RequirePackage[title,titletoc]{appendix}

\hypersetup{
    colorlinks=true,
    linkcolor=blue,
    filecolor=magenta,      
    urlcolor=cyan,
    pdftitle={PDF Title}, % put here the pdf title (different thing than document title)
    bookmarks=true,
    pdfpagemode=FullScreen,
    pdfauthor=%Put here your name
}
 
% Put all your images in the imagesw directory and treat that as a base path
\graphicspath{ {./images/} }

% set paper size and margins, right is smaller cause it seems like it is adding some margin to that
\geometry{
  a4paper,
  left=40mm,
  right=30mm,
  top=40mm,
  bottom=40mm
}



\pagestyle{fancy}
\fancyhf{}
\fancyhead[C]{Header title} % Put here what you want to see in the header of all pages

\fancypagestyle{empty}{
 \fancyhf{}
 \fancyhead[C]{Header title} % same thing again, it's beacause of different page types for title pages and stuff
}

\fancypagestyle{plain}{}


% just adds the "Page 2 of 20" on the bottom
\rfoot{Page \thepage \hspace{1pt} of \pageref*{LastPage}}



% Fill that, title and author are obvious, date is automaticaly set to compilation date
\title{Document title}
\author{Author}
\date{\today}


% Change the titles of those sections to better ones than defaults
\renewcommand{\contentsname}{Table of contents}
\renewcommand{\listfigurename}{List of figures}
\renewcommand{\listtablename}{List of tables}

% For table word wrapping
\emergencystretch=1em



% All the stuff below are for code formatting, keep it if oyu want, delete if not
% There are two styles defined here: 
% python - nice syntax highliting and none - white text on dark background
% to change the style use command \lstset{style=styleYouWant}, by default it's none

\newcommand{\hit}[1]{\textbf{\textcolor{ForestGreen}{#1}}}
\newcommand{\miss}[1]{\textbf{\textcolor{Red}{#1}}}


\definecolor{mygreen}{rgb}{0,0.6,0}
\definecolor{mygray}{rgb}{0.5,0.5,0.5}
\definecolor{bg}{rgb}{0.25,0.25,0.25}
\definecolor{mymauve}{rgb}{0.58,0,0.82}

\lstdefinestyle{python}{ 
  backgroundcolor=\color{bg},   % choose the background color; you must add \usepackage{color} or \usepackage{xcolor}; should come as last argument
  basicstyle=\footnotesize\color{white},        % the size of the fonts that are used for the code
  breakatwhitespace=false,         % sets if automatic breaks should only happen at whitespace
  breaklines=true,                 % sets automatic line breaking
  captionpos=b,                    % sets the caption-position to bottom
  commentstyle=\color{mygreen},    % comment style
  deletekeywords={...},            % if you want to delete keywords from the given language
  escapeinside={\%*}{*)},          % if you want to add LaTeX within your code
  extendedchars=true,              % lets you use non-ASCII characters; for 8-bits encodings only, does not work with UTF-8
  firstnumber=1,                   % start line enumeration with line 1000
  frame=single,	                   % adds a frame around the code
  keepspaces=true,                 % keeps spaces in text, useful for keeping indentation of code (possibly needs columns=flexible)
  keywordstyle=\color{orange},       % keyword style
  language=Python,                 % the language of the code
  morekeywords={*,...},            % if you want to add more keywords to the set
  numbers=left,                    % where to put the line-numbers; possible values are (none, left, right)
  numbersep=5pt,                   % how far the line-numbers are from the code
  numberstyle=\tiny\color{mygray}, % the style that is used for the line-numbers
  rulecolor=\color{black},         % if not set, the frame-color may be changed on line-breaks within not-black text (e.g. comments (green here))
  showspaces=false,                % show spaces everywhere adding particular underscores; it overrides 'showstringspaces'
  showstringspaces=false,          % underline spaces within strings only
  showtabs=false,                  % show tabs within strings adding particular underscores
  stepnumber=2,                    % the step between two line-numbers. If it's 1, each line will be numbered
  stringstyle=\color{ForestGreen},     % string literal style
  tabsize=2,	                   % sets default tabsize to 2 spaces
  identifierstyle=\color{white}
}

\lstdefinestyle{none}{ 
  backgroundcolor=\color{bg},   % choose the background color; you must add \usepackage{color} or \usepackage{xcolor}; should come as last argument
  basicstyle=\footnotesize\color{white},        % the size of the fonts that are used for the code
  breakatwhitespace=false,         % sets if automatic breaks should only happen at whitespace
  breaklines=true,                 % sets automatic line breaking
  captionpos=b,                    % sets the caption-position to bottom
  commentstyle=\color{white},    % comment style
  deletekeywords={...},            % if you want to delete keywords from the given language
  escapeinside={\%*}{*)},          % if you want to add LaTeX within your code
  extendedchars=true,              % lets you use non-ASCII characters; for 8-bits encodings only, does not work with UTF-8
  firstnumber=1,                   % start line enumeration with line 1000
  frame=single,	                   % adds a frame around the code
  keepspaces=true,                 % keeps spaces in text, useful for keeping indentation of code (possibly needs columns=flexible)
  keywordstyle=\color{white},       % keyword style
  morekeywords={*,...},            % if you want to add more keywords to the set
  numbers=left,                    % where to put the line-numbers; possible values are (none, left, right)
  numbersep=5pt,                   % how far the line-numbers are from the code
  numberstyle=\tiny\color{mygray}, % the style that is used for the line-numbers
  rulecolor=\color{black},         % if not set, the frame-color may be changed on line-breaks within not-black text (e.g. comments (green here))
  showspaces=false,                % show spaces everywhere adding particular underscores; it overrides 'showstringspaces'
  showstringspaces=false,          % underline spaces within strings only
  showtabs=false,                  % show tabs within strings adding particular underscores
  stepnumber=2,                    % the step between two line-numbers. If it's 1, each line will be numbered
  stringstyle=\color{white},     % string literal style
  tabsize=2,	                   % sets default tabsize to 2 spaces
  identifierstyle=\color{white},
  language=C
}

\lstset{style=none}